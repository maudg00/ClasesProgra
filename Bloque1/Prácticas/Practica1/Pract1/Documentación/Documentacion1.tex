\documentclass[conference]{IEEEtran}
\usepackage{cite}
\usepackage{amsmath,amssymb,amsfonts}
\usepackage{algorithmic}
\usepackage{graphicx}
\usepackage{textcomp}
\usepackage{xcolor}
\usepackage{color}
\usepackage{listings}
\usepackage{hyperref}
\usepackage[spanish]{babel}
\def\BibTeX{{\rm B\kern-.05em{\sc i\kern-.025em b}\kern-.08em
    T\kern-.1667em\lower.7ex\hbox{E}\kern-.125emX}} 
\definecolor{codegreen}{rgb}{0,0.6,0}
\definecolor{codegray}{rgb}{0.5,0.5,0.5}
\definecolor{codepurple}{rgb}{0.58,0,0.82}
\definecolor{backcolour}{rgb}{0.95,0.95,0.92}
\definecolor{terminalbg}{rgb}{00.01,0.01,0.01}
\definecolor{terminaltext}{rgb}{0.95,0.95,0.95}
\definecolor{codedark}{rgb}{0.8, 0.2, 0.5}
\definecolor{codered}{rgb}{1,0.2,0.2}

\lstdefinestyle{mystyle}{
    backgroundcolor=\color{backcolour},   
    commentstyle=\color{codegreen},
    keywordstyle=\color{magenta},
    numberstyle=\tiny\color{codegray},
    stringstyle=\color{codepurple},
    emphstyle=\color{codedark},
    tagstyle =\color{codered},
    basicstyle=\ttfamily,
    breakatwhitespace=true,         
    breaklines=true,  
    breakindent=0em,
    captionpos=b,                    
    keepspaces=true,                 
    numbers=left,                    
    numbersep=5pt,                  
    showspaces=false,                
    showstringspaces=false,
    showtabs=false,      
    tabsize=2,
}

\lstdefinestyle{terminal}{
    backgroundcolor=\color{terminalbg},
    basicstyle\color{terminaltext},   
    commentstyle=\color{terminaltext},
    keywordstyle=\color{terminaltext},
    numberstyle=\tiny\color{codegray},
    stringstyle=\color{terminaltext},
    basicstyle=\ttfamily,
    breakatwhitespace=true,         
    breaklines=true,  
    breakindent=0em,
    captionpos=b,                    
    keepspaces=true,                 
    numbers=left,                    
    numbersep=5pt,                  
    showspaces=false,                
    showstringspaces=false,
    showtabs=false,                  
    tabsize=2,
}

\begin{document}

\title{Práctica 1. Programa para calcular sumas consecutivas y cambios de base
\thanks{Muchas gracias a Mauricio de Garay}
}

\author{\IEEEauthorblockN{1\textsuperscript{st} José Luis Aguilar}
\IEEEauthorblockA{\textit{Departamento de aprender a programar} \\
\textit{Universidad MdG}\\
Ciudad de México, México \\
aguilarch.joseluis@gmail.com}
}

\maketitle

\begin{abstract}
Este documento presenta la documentación de un programa en C que pueda calcular
las sumas consecutivas dado un número, y también una herramienta para el cambio
de base decimal a cualquier otra. Se muestran además los procesos que sigue el
programa y las herramientas de diseño usadas para la solución. 
\end{abstract}

\begin{IEEEkeywords}
programación, Clang, introducción, cambio de base, sumas de números consecutivos.
\end{IEEEkeywords}

\section{Descripción del problema}
Se requiere de un programa que tenga la capacidad de llevar a cabo tres funciones 
manejadas a través de un menú:
\begin{enumerate}
    \item Recibir un número entero positivo del usuario y encontrar todas las series de 3 o más números consecutivos que sumados den como resultado este valor.
    \item Recibir un número entero positivo del usuario, convertirlo a 3 bases (de igual manera leídas por el usuario) entre base 2 y 16.
    \item Salir del programa. 
\end{enumerate}
Además, antes de mostrar el menú, se muestra el nombre del creador, título del programa y fecha. 

\section{Comportamiento detallado}



\section{Procesos}
A continuación se muestra el proceso detallado del programa, separado en pasos para facilitar su lectura:
\begin{itemize}
    \item 
\end{itemize}

\section{Alcances y limitaciones}
El programa puede:
\begin{itemize}
    \item Convertir de base decimal a cualquier base numérica entre 2 y 16.
    \item Mostrar todas las series de 3 o más sumas de números consecutivos.
    \item Valida que los valores introducidos sean números enteros positivos.
    \item Valida que no puedan ingresarse opciones inexistentes.
\end{itemize}
Las limitaciones del programa son las siguientes:
\begin{itemize}
    \item Solamente puede tomar números enteros positivos, no convierte otros tipos de datos.
    \item No utiliza arreglos de datos, ni recursión, ni variables globales. 
    \item Si el programa detecta un valor válido, regresa al menú y no directamente a la función que estaba usándose.
\end{itemize}
\section{Diseño de pantalla}
El programa muestra una descripción al inicio, con la siguiente forma\footnote{Se omiten los acentos en las pantallas del programa aquí descritas, pero en el ejecutable sí se observan.}:
\begin{lstlisting}[style=terminal]
-------------------------------------------------------------------------------

Este programa fue desarrollado por Jose Luis Aguilar el 23/12/2020.
Presione [Enter] para continuar.
    
-------------------------------------------------------------------------------
\end{lstlisting}
Que después menciona que se muestra un menú. El menú se ve de la forma siguiente:
\begin{lstlisting}[style=terminal]
-------------------------------------------------------------------------------
                Menu
               ------
Escoge una opcion escribiendo el numero de la accion que desees tomar:

    1. Cambio de base a un numero
    2. Calcular sumas consecutivas de un numero
    3. Salir
-------------------------------------------------------------------------------
\end{lstlisting}
Cada opción muestra la acción que toma. A continuación se muestran en orden las distintas opciones:
\begin{lstlisting}[style=terminal]
Introduzca el numero entero positivo a convertir: 10423
Introduzca la base (entre 2 y 16) a la que desea convertir: 12
El numero en base 12 es: 6047
Presione [Enter] para continuar.
\end{lstlisting}
Cada distinta opción elegida regresa nuevamente al menú.
\begin{lstlisting}[style=terminal]
Introduzca el numero entero positivo a mostrar sus sumas: 3456
3 sumas consecutivas: 
1151 + 1152 + 1153 
9 sumas consecutivas: 
380 + 381 + 382 + 383 + 384 + 385 + 386 + 387 + 388 
27 sumas consecutivas: 
115 + 116 + 117 + 118 + 119 + 120 + 121 + 122 + 123 + 124 + 125 + 126 + 127 + 128 + 129 + 130 + 131 + 132 + 133 + 134 + 135 + 136 + 137 + 138 + 139 + 140 + 141 
Presione [Enter] para continuar
\end{lstlisting}
Por último, la opción de salir del programa pide al usuario que confirme su decisión:
\begin{lstlisting}[style=terminal]
----------------------------------------------

                Desea salir?

               1 = si, 0 = no.

----------------------------------------------
\end{lstlisting}
Debido al tamaño usual de los displays en la terminal de 80 caracteres,
el diseño del menú y las descripciones se hizo de 80 caracteres de ancho.
\section{Diseño de solución}
Antes de mostrar el pseudocódigo y los algoritmos utilizados en el desarrollo 
del programa, conviene comprender el fundamento matemático sobre el cual se basan
ambas soluciones. A continuación se muestra el fundamento de cada uno, donde se mostrará
un pequeño pseudocódigo correspondiente al final de cada algoritmo.
    \subsection{Cambio de base}
        Una base numérica se basa en la notación dada por la siguiente ecuación:
        \begin{equation*}
            X_b = \sum_{n=-\infty}^{d}a_nb^n
        \end{equation*}
        Donde $X_b$ es el número escrito en la base $b$, $a_n$ es el múltiplo
        de cada una de las potencias de la base $b$, donde suele representarse con un símbolo
        comprendido en el conjunto $a_n \in \{0,...,b-1\}$\footnote{Nos remitimos solamente a bases enteras, ya que aunque existen aquellas que no son enteras, suelen no ser utilizadas fuera de ámbitos de investigación matemática.},
        $d$ es el dígito correspondiente. Si se asume que solamente se trabaja con enteros, entonces se reduce a la siguiente expresión:
        \begin{equation}
            X_b = \sum_{n=0}^{d}a_nb^n
        \end{equation}
        Por lo que para representar al número, se requieren $n+1$ dígitos. El número de dígitos puede encontrarse usando
        la siguiente propiedad de los logaritmos:
        \begin{equation*}
            \log_b(x)=\frac{\log(x)}{\log(b)}
        \end{equation*}
        Entonces, considerando que $\lfloor\log_b(x)\rfloor$ es el número de dígitos
        necesarios para expresar a $x$ en la base $b$, puede encontrarse el valor de cada uno con un sencillo algoritmo descrito en el siguiente pseudocódigo:
        \begin{lstlisting}[style=mystyle,    language=C]
cambio de base (int numero, int base)
    ValidarEnteroPositivo(numero);
    ValidarEnteroPositivo(base);
    int digito = log(numero)/log(base);
    int multiplo = 0;
    int residuo = 0;

    while(numero>0)
        if(residuo>=numero)
            residuo=residuo + power(base,digito);
            multiplo=multiplo+1;
        else
            // Se resta porque al contar existe un overshoot.
            residuo = residuo - power(base,digito);
            multiplo = multiplo-1;
            digito = digito-1;
            numero = numero-residuo;
            PRINT("_entero_",multiplo);
            residuo=0;
            numero=0; 
        FIN
    FIN
    // Para considerar los que son multiplos de la base:
    for(entero i = 0, i<=digito,i++)
        print("0")
    FIN
        \end{lstlisting}
        Esto muestra un algoritmo relativamente sencillo, y la implementación en
        el programa es muy similar, solamente con la diferencia que considera también
        las entradas de números y formatos para hacerlo más legible. 
    \subsection{Sumas consecutivas}
        Para poder considerar las sumas consecutivas, se puede hacer un algoritmo
        relativamente sencillo. Consideremos primero la suma de números enteros
        positivos consecutivos más grande posible:
        \begin{equation*}
            X = \sum_{n=1}^N n
        \end{equation*}
        Se omite el cero porque si se considera, entonces cualquier serie pudiera
        ser más grande. Sin embargo, esta serie de números forzosamente es la más
        grande porque considera a todos los enésimos enteros. Esto impone un límite
        superior sobre las iteraciones que puede realizar el programa, que puede
        expresarse de una forma más sencilla, descrita en la siguiente ecuación\footnote{En los anexos se hace una breve demostración por si el lector no se encuentra familiarizado.}:
        \begin{equation}
            \label{Gauss}
            X = \frac{n(n+1)}{2}
        \end{equation}
        Esta suma considera todos los enteros. Sin embargo, solamente son de interés
        en este programa las sumas que tengan tres o más enteros. Además, pueden
        considerarse dos casos distintos; cuando el número de términos consecutivos
        sea un número par, y cuando sea impar. Hecha esta consideración, entonces
        se pueden considerar los números de dos en dos casos, y se sigue respetando el 
        límite superior. Haciendo el cambio de variable $n=2k+1$ en \eqref{Gauss}:
        \begin{equation}
            X = \frac{(2k+1)(2k+2)}{2}=(2k+1)(k+1)
        \end{equation}
        Y esto es el límite superior tomando las sumas consecutivas de dos en dos. 
        Entonces vale la pena considerar en qué casos existen estas sumas consecutivas. 
        Esto ocurre para un número $X$ con un número de sumas consecutivas $n$ bajo los siguientes casos:
        \begin{equation*}
            \begin{cases}
                n\; \mathrm{par} &\; X\, \mathrm{mod}\, n\, \equiv\, \frac{n}{2} \\
                n\; \mathrm{impar} &\; X\, \mathrm{mod}\, n\, \equiv\, 0
            \end{cases}
        \end{equation*}
        Estas condiciones son relativamente sencillas de encontrar, y se cumplen
        siempre. Suponiendo primero que el número de términos sea impar, entonces
        puede pensarse en el número $X$ dividido entre el número de términos $n$,
        que forzosamente va a ser el término medio de las sumas consecutivas.
        Entonces la suma de $n$ términos puede ponerse de la siguiente forma:
        \begin{equation*}
            X = \frac{1}{n}\left[(X-\tfrac{n-1}{2})+\dots+X+\dots+(X+\tfrac{n-1}{2})\right]
        \end{equation*}
        Por el otro lado, si número de términos es par, y solamente se trabaja
        con enteros, entonces pueden tomarse solamente los cocientes enteros, y
        como la siguiente relación se cumple cuando el residuo es distinto a cero: 
        $\left\lfloor\tfrac{X}{n}\right\rfloor<\tfrac{X}{n}$
        Entonces también puede medirse todo con respecto a un dígito (en este caso,
        el número $\tfrac{n}{2}$). Siguiendo este criterio, entonces puede escribirse
        un número de la siguiente forma:
        \begin{equation*}
            X = \sum_{i=1}^n\left(\left\lfloor\frac{X}{n}\right\rfloor-\frac{n}{2}+i+1\right)
        \end{equation*}
        Tomando en cuenta estas dos relaciones, entonces puede hacerse un sencillo
        algoritmo para poder calcular las sumas consecutivas dependiendo si el 
        número es par o no. A continuación se muestra el pseudocódigo:
        \begin{lstlisting}[style=mystyle]
void SumasConsecutivas(int *X)
    n = 1;
    while((2*n+1)*(n+1)<= *X) 
        SumasImpares(*X, 2*n+1); 
        SumasPares(*X, 2*(n+1));
        n++;
    FIN

void SumasImpares(int X, int N)
    if(X % N == 0) // % es la operacion modulo.
    
        int offset = (N-1)/2;
        int ValorInicial = X/N-offset; 
        int sumas = ValorInicial;

        /* El primer termino esta fuera del loop, 
        con el proposito de que al imprimir los 
        valores de la suma, no salga un termino de mas. */

        print("_num_ sumas consecutivas: _num_ ", N, sumas);

        for(int i = 1, i < N, i++)
            sumas = ValorInicial+i;
            printf("+ _num_ ", sumas);
        FIN
        print("_newline_");
    FIN

void SumasPares(entero X, entero N)
    if(X%N==N/2 AND X>=N*(N+1)/2)

        int offset = N/2-1;
        int ValorInicial = X/N - offset; 
        int sumas = ValorInicial;
        print("_num_ sumas consecutivas: _num_ ", N, sumas);
        for(int i = 1, i < N, i++)
            sumas = ValorInicial+i;
            printf("+ _num_ ", sumas);
        FIN
        printf("_newline_");
    FIN
        \end{lstlisting}
        
        
\end{document}
