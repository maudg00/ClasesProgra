\documentclass[conference]{IEEEtran}
% \IEEEoverridecommandlockouts
% The preceding line is only needed to identify funding in the first footnote. If that is unneeded, please comment it out.
\usepackage{cite}
\usepackage{amsmath,amssymb,amsfonts}
\usepackage{algorithmic}
\usepackage{graphicx}
\usepackage{textcomp}
\usepackage{xcolor}
\usepackage{color}
\usepackage{listings}
\usepackage[spanish]{babel}
\def\BibTeX{{\rm B\kern-.05em{\sc i\kern-.025em b}\kern-.08em
    T\kern-.1667em\lower.7ex\hbox{E}\kern-.125emX}}
    

\definecolor{codegreen}{rgb}{0,0.6,0}
\definecolor{codegray}{rgb}{0.5,0.5,0.5}
\definecolor{codepurple}{rgb}{0.58,0,0.82}
\definecolor{backcolour}{rgb}{0.95,0.95,0.92}
\definecolor{terminalbg}{rgb}{00.01,0.01,0.01}
\definecolor{terminaltext}{rgb}{0.95,0.95,0.95}

\lstdefinestyle{mystyle}{
    backgroundcolor=\color{backcolour},   
    commentstyle=\color{codegreen},
    keywordstyle=\color{magenta},
    numberstyle=\tiny\color{codegray},
    stringstyle=\color{codepurple},
    basicstyle=\ttfamily,
    breakatwhitespace=true,         
    breaklines=true,  
    breakindent=0em,
    captionpos=b,                    
    keepspaces=true,                 
    numbers=left,                    
    numbersep=5pt,                  
    showspaces=false,                
    showstringspaces=false,
    showtabs=false,      
    tabsize=2
}

\lstdefinestyle{terminal}{
    backgroundcolor=\color{terminalbg},
    basicstyle\color{terminaltext},   
    commentstyle=\color{terminaltext},
    keywordstyle=\color{terminaltext},
    numberstyle=\tiny\color{codegray},
    stringstyle=\color{terminaltext},
    basicstyle=\ttfamily,
    breakatwhitespace=true,         
    breaklines=true,  
    breakindent=0em,
    captionpos=b,                    
    keepspaces=true,                 
    numbers=left,                    
    numbersep=5pt,                  
    showspaces=false,                
    showstringspaces=false,
    showtabs=false,                  
    tabsize=2,
}

\begin{document}

\title{Práctica 0. Programa para calcular las raíces de un polinomio de segundo grado
\thanks{Muchas gracias a Mauricio de Garay}
}

\author{\IEEEauthorblockN{1\textsuperscript{st} José Luis Aguilar}
\IEEEauthorblockA{\textit{Departamento de aprender a programar} \\
\textit{Universidad MdG}\\
Ciudad de México, México \\
aguilarch.joseluis@gmail.com}
}

\maketitle

\begin{abstract}
Este documento presenta la documentación de un programa en C para encontrar las soluciones reales de una ecuación de segundo grado.
Se presentan los requisitos del programa, alcances y limitaciones, así como herramientas de diseño de la solución utilizadas.
\end{abstract}

\begin{IEEEkeywords}
programación, Clang, introducción, ecuaciones de segundo grado.
\end{IEEEkeywords}

\section{Descripción del problema}
Se requiere un programa que al ingresarle los coeficientes de un polinomio de segundo grado de una sola variable en su forma canónica, devuelva las raíces correspondientes. La forma general del polinomio es dado por la siguiente ecuación:
\begin{equation*}
    ax^2+bx+c = 0
\end{equation*}
Y sus soluciones serán denotadas como $x_1$ y $x_2$, independientemente de si sean iguales o distintas las soluciones.
Las entradas y salidas serán todas por la terminal, y no se utiliza otra interfaz gráfica para el usuario.

\section{Comportamiento detallado}
El programa solicita los coeficientes del polinomio de segundo grado. Posteriormente valida que todos los coeficientes sean mayores a cero, y procede a calcular las raíces. Si las raíces son complejas, el programa devuelve las raíces iguales a cero, y muestra una advertencia que no representa a la solución verdadera.

Las entradas que necesita el programa son los coeficientes cuadrático, lineal e independiente del polinomio– $a$, $b$ y $c$ respectivamente. 

La salida que devuelve son las raíces, denotadas como $x_1$ y $x_2$.

\section{Procesos}
A continuación se muestra el proceso detallado del programa, separado en pasos para facilitar su lectura:
\begin{itemize}
    \item \textbf{Mensaje introductorio:} se menciona que el programa encuentra las raíces del polinomio de segundo grado.
    \item \textbf{Entrada de datos:} se solicitan en orden los coeficientes de los términos cuadrático, lineal e independiente en ese orden. El programa verifica que todos los coeficientes introducidos sean escritos en su forma numérica\footnote{No se pueden introducir datos de otra manera, (e.g. "DOS" no es una entrada válida, pero "2" sí} y que sean mayores a 0.
    \item \textbf{Cálculo de raíces:} Una vez ingresados los datos, se procede a calcular las raíces, siguiendo los siguientes pasos:
    \begin{itemize}
        \item Se calcula el discriminante de la ecuación de segundo grado, definido de la siguiente forma: $$\Delta=b^2-4ac$$
        \item Si el discriminante es mayor que cero, se procede a calcular las raíces. Si es menor a cero, se devuelven ceros como las soluciones, pues solamente calcula las raíces reales.
        \item Para calcular las raíces, se utiliza la fórmula general para las ráices de segundo grado, descrito en la siguiente ecuación: $$\frac{-b\pm\sqrt{\Delta}}{2a}$$
    \end{itemize}
    \item \textbf{Mostrar resultados:} se despliegan las raíces $x_1$ y $x_2$.
\end{itemize}

\section{Alcances y limitaciones}
    El programa puede tomar como entradas a coeficientes racionales y mayores a cero. No recibe como argumentos a números escritos con letra, ni a números negativos. Solamente puede calcular raíces reales, y no puede encontrar raíces complejas. 


\section{Diseño de pantalla}
    Las interacciones ocurren todas en la terminal. Por lo tanto, el diseño de pantalla es sumamente sencillo. A continuación se muestra un escenario en el que se use el programa:

\begin{lstlisting}[style=terminal]
$pract0.c
Este programa calcula las raices de un polinomio de segundo grado
descritos en su forma canonica. Su forma canonica es la siguiente:
a*x^2 + b*x + c = 0
    
Inserte el coeficiente del termino cuadratico:
3
Inserte el coeficiente del termino lineal:
x
El valor otorgado no es un numero, o es menor a 0.            
Intente de nuevo.
10000
Inserte el coeficiente del termino independiente:
2
Las raices para el polinomio:
3.00 x^2 + 10000.00 x + 2.00 = 0
Son las siguientes:
x1 = -0.000200
x2 = -3333.333133
\end{lstlisting}

\section{Diseño de solución}
    A continuación se muestra el pseudocódigo desarrollado para el programa. Anexo
    a este documento, se encuentra una prueba de escritorio para observar el
    comportamiento del programa en una situación ficticia. Se eliminan todos los
    acentos del pseudocódigo para hacer el render, pero se toma en cuenta que
    contiene errores ortográficos.

    \subsection{Pseudocódigo}
\begin{lstlisting}[style=mystyle]
MAIN
float a = 0, b = 0, c = 0, x1 = 0, x2 = 0;

Print("Este programa calcula las raices de un polinomio de segundo grado \n");
Print("descritos en su forma canonica. Su forma cannica es la siguiente: \n");
Print("a*x^2 + b*x + c = 0 \n");

PedirValor(&a, "Inserte el coeficiente del termino cuadratico:\n");
PedirValor(&b, "Inserte el coeficiente del termino lineal:\n");
PedirValor(&c, "Inserte el coeficiente del termino independiente:\n");
CalcularRaices(a, b, c, &x1, &x2);
Desplegar(a, b, c, x1, x2);

// Funciones auxiliares

Void PedirValor(float *a, string description); 
	
	int flag = 1;
	Print(description);	
	while (flag == 0) then:
		Read(a);
		if *a > 0
			flag = 1;
		else 
			Print("El valor otorgado no es un numero, o es menor a 0. Intente de nuevo.\n");
	FIN

Void CalcularRaices(float a, float b, float c, float *x1, float *x2)
	float discriminante = power(b,2) - 4*a*c;
	
	if discriminante < 0
		Print("El valor del discriminante del polinomio es menor a cero.\n");
		Print("El programa no permite resolver polinomios con raices complejas.\n
			El programa devolvera ceros. \n");
	else 
		*x1 = (-b + sqrt(discriminante))/(2*a);
		*x2 = (-b - sqrt(discriminante))/(2*a);


Void Desplegar(float a, float b, float c, float x1, float x2)
	Print(
		"Las raices para el polinomio:\n
		__float__x^2 + __float__x+__float__=0\n
	          Son las siguientes:\n 
		x1 = __int__\n 
		x2 = __int__", 
		a, b, c, x1, x2
		);
\end{lstlisting}


\end{document}
